\documentclass{article}
\usepackage{graphicx} % Required for inserting images
\usepackage{tcolorbox}
\usepackage{xcolor}
\usepackage{tcolorbox}
\usepackage{etoolbox}

\usepackage{amsfonts,amsmath,amssymb}
\title{Exos de Colles HX4}
\author{Classe de MP2I LLG.}
\date{January 2026}
\newcounter{exo}
\newcommand{\beginexo}[2][]{
  \ifstrequal{#1}{cours}{
    \def\exocolor{brown}
    \def\exotitle{Question de cours}
  }{
    \refstepcounter{exo}
    \def\exocolor{black}
    \def\exotitle{Exercice \theexo}}

  \begin{tcolorbox}[
    colback=\exocolor!5,
    colframe=\exocolor,
    title=\exotitle,
    fonttitle=\bfseries,
    boxrule=0.8pt,
    sharp corners
  ]
  \textit{#2}
  \end{tcolorbox}
}

\begin{document}

\maketitle

\section{Introduction}
Le fonctionnement est le suivant : vous utilisez la commande beginexo[flag(optional)]\{Enonce de l'exercice\} pour ajouter un exercice. Le code couleur est simple : marron pour une question de cours et noir pour un exercice. Pour une question de cours mettre flag à cours. Have fun et bon exo solving. J'ai perdu.
\section{Exercices}
\beginexo[cours]{Enoncer et prouver le critère de d'Alembert pour les suites réelles.}
\beginexo{Soit $(A_n)$ une suite d'ouverts denses de $\mathbb{R}$. Montrer que $\cap_{n\in\mathbb{N} }A_n$ est dense dans $\mathbb{R}$.}
\beginexo{Soit $(x_n)_{n\in\mathbb{N}} \in \mathbb{R}^{\mathbb{N}}$, on pose pour tout $n\in \mathbb{N}$ : $y_n=x_{n-1}+2x_n$. \newline Montrer que $(x_n)$ converge $\iff (y_n)$ converge.}

\beginexo{Pour tout $n\in\mathbb{N}$ on pose $q_n$ le plus petit entier premier positif tel que $q_n$ ne divise pas $n$. Montrer que:
\begin{itemize} 
          \item $\frac{q_n}n \to 0$ dans un premier temps.
          \item $\frac{q_n}{\sqrt{n}} \to 0$. 
\end{itemize}}
\end{document}
